% vim: fo=aw2tq tw=100 spell
\section{Design and Implementation}

\subsection{Keypad handler}

The keypad is a slightly awkward device for trying to get a single value --- it interrupts the whole 
time that the key is being pressed.  When implementing the keypad handler, this was the main problem 
I had to overcome.

My solution was to create a kind of lockout for that particular interrupt handler.  The result is a 
variable that acts as an ``already handled'' flag for keypad interrupts.  When a keypad interrupt 
happens, the flag is checked --- if it's non-zero, then the interrupt gets effectively ignored.  If 
it's zero, the interrupt handler executes as normal.  At the end of the interrupt handler, there are 
a few NOP instructions before setting the lockout variable back to zero and returning from the 
interrupt, but \emph{after} the interrupts are re-enabled.  The idea is that if the key is still 
pressed, it will interrupt during those NOPs, before the flag is cleared, and those interrupts will 
get ignored, but on the very last interrupt it will complete the handler.  This completely 
eliminates the possibility of accidental repeats.

For comparison, the alternative solution was to insert a delay after handling the keypress to allow 
time for the key to be released.  Unfortunately, that method was less than ideal, since if the delay 
was too long it would result in the system seeming unresponsive, and if too short then repeated 
interrupts would occur.  I felt it was better to maintain a direct responsiveness between action and 
reaction.

\subsection{Note lookup tables}
\label{notelookuptables}

Lookup tables are needed to contain the PRT value, divisor and note name (including octave) for 
every MIDI note.  The note names are in a separate lookup table to the other data as having an LCD 
information display was a lower priority than working sound playback and was implemented much later.

% TODO: expand on comparison of formats
The most efficient way to use a lookup value for a table is to multiply it by a power of two (left 
shifting), so blocks of data in the lookup table are also going to be $2^n$ bytes long.  However, 
the data for note playback is 3 bytes (a 16-bit PRT value plus an 8-bit divisor), so I was faced 
with the choice of having two lookup tables for the playback data or wasting $\frac{1}{4}$ of the 
space used by the table.  After writing the necessary assembler for reading from both possible 
formats and analysing their execution times, I found that the single table method resulted in faster 
code.  Combined with the fact I was not under pressure to conserve memory, I chose the single table 
method.

% TODO: expand on the result sorting
The Python script\footnote{"tools/pitchtable.py" in the submitted source code} I wrote to generate 
the lookup tables\footnote{"note\_lookup.s"} is based on the concept outlined in \ref{wavetables} 
for creating new frequencies from a single sample by modifying the sample rate and divisor.  PRT and 
divisor values are calculated for the frequency information in the pitch 
table\footnote{"tools/pitchtable.csv" - octave, MIDI note number, note name, frequency}.

The first step is calculating the PRT value for every divisor, $D$ from 1 to 255, first by 
calculating the sample rate using

\[R_{target} = \frac{F_{target} \times R_{source}}{F_{source} \times D}\]

(rearranged from a similar equation in \ref{wavetables}) and then converting this to a PRT value 
using

\[PRT = round\big(\frac{F_{CPU}}{D_{PRT}\times{}R_{target}}\big)\]

where $F_{CPU}$ is the CPU frequency (6,144,000) and $D_{PRT}$ is the PRT divisor (20).  Because the 
result has been rounded to the nearest integer, the effective frequency will be different to the 
desired frequency.  The effective frequency is calculated by using the $PRT$ and $R_{target}$ 
calculations in reverse:

\[F_{effective} = \frac{F_{source}\times{}D\times{}\frac{F_{CPU}}{D_{PRT}\times{}PRT}}{R_{source}}\]

The error, E is calculated with the following, giving a value in the range 0--1:

\[E = \frac{|F_{effective} - F_{target}|}{F_{target}}\]

% TODO: to joke or not to joke?
For each note, the results for all the divisors are collected together, discarding those where the 
sample rate is higher than 8kHz (a sensible limit given the CPU speed), and sorted in reverse order 
by $R^{1-E}$.  The idea is that the basic ordering aims for a high sample rate, but any kind of 
significant error greatly diminishes the value of the result.  The first result is taken from the 
sorted list, giving the PRT and divisor values the algorithm decides best represent the frequency.  
While almost seeming like mathematical voodoo, the worst error rate for any note in the resulting 
table is $1.13\%$, with an average sample rate of just under 7kHz.  Other sorting methods I tried 
had varied results, generally creating high sample rates with high error margins, or low error 
margins with bad sample rates.

The format of the resulting lookup table is:
\begin{center}
\begin{tabular}{c | c | c | c | c}
Byte & 0 & 1 & 2 & 3 \\
\hline
Usage & PRT low & PRT high & Divisor & 00h \\
\end{tabular}
\end{center}

The script also produces a second table which contains a 4-character representation of the note 
created from the octave number and note name.
