% vim: fo=aw2tq tw=100 spell
\section{Design and Implementation}

\subsection{Audio output}
\label{sec:design:audio_output}

% TODO: sound/music theory somewhere?
The output device was jointly designed with my lab partner.  The most important property of any 
output device developed was obviously the ability to output a waveform that would audibly represent 
the intended musical note (and instrument).  However, since the network data also includes volume 
information, this needed to be taken into account.  This could be achieved in two ways:
\begin{enumerate}
\item Implementing a volume scaling function in software
\item Scaling the volume with more programmable hardware
\end{enumerate}
The first option has the advantage of giving the simplest, and therefore easiest, hardware to 
design.  Unfortunately, scaling the volume of a waveform during playback would involve modifying the 
amplitude of every step of a waveform with multiplication.  This would result in playback being a 
very processor-intensive task, which is fairly impractical for the speed and simplicity of our CPU.  
On the other hand, implementing volume scaling in hardware removes this burden from the CPU, freeing 
up the processing time for other tasks.

We chose to implement the volume scaling in hardware, and gave the hardware a very simple interface 
--- an 8-bit ``waveform'' input and an 8-bit ``volume'' input.  This would later make outputting a 
waveform a very fast software task --- one I/O operation per sample in the waveform, and one I/O 
operation for every volume value received.

The conversion from the two 8-bit values to the final audio signal can be described as a two-stage 
digital-to-analogue converter (DAC) with a final amplification stage.  A simple circuit of a DAC and 
inverting op-amp will give an output voltage between ground and $-V_{ref}$ proportional to the 
digital input (where $V_{ref}$ is the reference voltage of the DAC).  We decided to use this concept 
to develop a two-stage DAC consisting of two such circuits, where the output of the first became the 
$V_{ref}$ of the second.  The first becomes the ``volume'' DAC, effectively pre-scaling any signal 
output by the second, which would receive the waveform sample values.  Note that this design does 
not prevent the use of software scaling if desired, but gives the attractive possibility of avoiding 
it.

To interface with the SBC, the DAC inputs were connected to ports A and B of the parallel I/O 
controller.  Though the network data contained 7-bit volume values, and the hardware needed 8-bit 
values, we decided not to ``left-shift'' the data in hardware (wiring parallel I/O data pins 0--6 to 
DAC data pins 1--7) to maintain flexibility of the hardware.  (In my solution the volume values are 
left-shifted in software.)  The result of this was that existing I/O ports on the SBC were used, 
removing the need to create extra hardware for chip selection based on I/O address, and the DAC 
operation was simple because the parallel I/O is latched (data stays on the outputs between writes).

There are a few things worth noting in the circuit diagram for the audio output device (appendix 
\ref{appendix:circuit_diagram}):

\subsubsection{$V_{ref}$ Voltage Divider}

In the top-right of the circuit diagram there is a voltage divider between the $V_{ref}$ input of 
the volume DAC and $+12V$ (R1, R2 and C2).  It supplies a $??V$ potential, made very smooth by the 
$22\mu{}F$ capacitor --- the DAC is very sensitive to noise on $V_{ref}$.

\subsection{Keypad handler}
\label{sec:design:keypad_handler}

The keypad is a slightly awkward device for trying to get a single value --- it interrupts the whole 
time that the key is being pressed.  When implementing the keypad handler, this was the main problem 
I had to overcome.

My solution was to create a kind of lockout for that particular interrupt handler.  The result is a 
variable that acts as an ``already handled'' flag for keypad interrupts.  When a keypad interrupt 
happens, the flag is checked --- if it's non-zero, then the interrupt gets effectively ignored.  If 
it's zero, the interrupt handler executes as normal.  At the end of the interrupt handler, there are 
a few NOP instructions before setting the lockout variable back to zero and returning from the 
interrupt, but \emph{after} the interrupts are re-enabled.  The idea is that if the key is still 
pressed, it will interrupt during those NOPs, before the flag is cleared, and those interrupts will 
get ignored, but on the very last interrupt it will complete the handler.  This completely 
eliminates the possibility of accidental repeats.

For comparison, the alternative solution was to insert a delay after handling the keypress to allow 
time for the key to be released.  I found this method to be less than ideal, since if the delay is 
too long it can result in the system seeming unresponsive, and if too short then repeated interrupts 
can occur.  I felt it was better to maintain a direct response between action and reaction.

\subsection{Note lookup tables}
\label{notelookuptables}

Lookup tables are needed to contain the PRT value, divisor and note name (including octave) for 
every MIDI note.  The note names are in a separate lookup table to the other data as having an LCD 
information display was a lower priority than working sound playback and was implemented much later.

% TODO: expand on comparison of formats
The most efficient way to use a lookup value for a table is to multiply it by a power of two (left 
shifting), so blocks of data in the lookup table are also going to be $2^n$ bytes long.  However, 
the data for note playback is 3 bytes (a 16-bit PRT value plus an 8-bit divisor), so I was faced 
with the choice of having two lookup tables for the playback data or wasting $\frac{1}{4}$ of the 
space used by the table.  After writing the necessary assembler for reading from both possible 
formats and analysing their execution times, I found that the single table method resulted in faster 
code.  Combined with the fact I was not under pressure to conserve memory, I chose the single table 
method.

% TODO: expand on the result sorting
The Python script\footnote{"tools/pitchtable.py" in the submitted source code} I wrote to generate 
the lookup tables\footnote{"note\_lookup.s"} is based on the concept outlined in \ref{wavetables} 
for creating new frequencies from a single sample by modifying the sample rate and divisor.  PRT and 
divisor values are calculated for the frequency information in the pitch 
table\footnote{"tools/pitchtable.csv" - octave, MIDI note number, note name, frequency}.

The first step is calculating the PRT value for every divisor, $D$ from 1 to 255, first by 
calculating the sample rate using

\[R_{target} = \frac{F_{target} \times R_{source}}{F_{source} \times D}\]

(rearranged from a similar equation in \ref{wavetables}) and then converting this to a PRT value 
using

\[PRT = round\left(\frac{F_{CPU}}{D_{PRT}\times{}R_{target}}\right)\]

where $F_{CPU}$ is the CPU frequency (6,144,000) and $D_{PRT}$ is the PRT divisor (20).  Because the 
result has been rounded to the nearest integer, the effective frequency will be different to the 
desired frequency.  The effective frequency is calculated by using the $PRT$ and $R_{target}$ 
calculations in reverse:

\[F_{effective} = \frac{F_{source}\times{}D\times{}\frac{F_{CPU}}{D_{PRT}\times{}PRT}}{R_{source}}\]

The error, E is calculated with the following, giving a value in the range 0--1:

\[E = \frac{\left|F_{effective} - F_{target}\right|}{F_{target}}\]

% TODO: to joke or not to joke?
For each note, the results for all the divisors are collected together, discarding those where the 
sample rate is higher than 8kHz (a sensible limit given the CPU speed), and sorted in reverse order 
by $R^{1-E}$.  The idea is that the basic ordering aims for a high sample rate, but any kind of 
significant error greatly diminishes the value of the result.  The first result is taken from the 
sorted list, giving the PRT and divisor values the algorithm decides best represent the frequency.  
While almost seeming like mathematical voodoo, the worst error rate for any note in the resulting 
table is $1.13\%$, with an average sample rate of just under 7kHz.  Other sorting methods I tried 
had varied results, generally creating high sample rates with high error margins, or low error 
margins with bad sample rates.

The format of the resulting lookup table is:
\begin{nowordcount}
\begin{center}
\begin{tabular}{c | c | c | c | c}
Byte & 0 & 1 & 2 & 3 \\
\hline
Usage & PRT low & PRT high & Divisor & 00h \\
\end{tabular}
\end{center}
\end{nowordcount}

The script also produces a second table which contains a 4-character representation of the note 
created from the octave number and note name.
