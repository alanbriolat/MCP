% vim: fo=aw2tq tw=100 spell
\section{Further Work}

\subsection*{Note lookup tables}
% TODO: expand on this, write something about *_print functions?
Since the major use of the note lookup tables is by the network handler, which uses all values of 
both tables, given a little more time it would make sense to combine the playback and display 
information into a single lookup table.  The new lookup table would contain blocks of 8 bytes, 
however the data only amounts to 7 bytes.  This would actually be useful, as the 8th byte would be 
set to 00h, which is the string terminator used by "lcd_print"\footnote{Line 48 of "lcd.s"}.  If the 
new lookup table format was

\begin{center}
\begin{tabular}{r | c | c | c | c | c | c | c | c}
Byte & 0 & 1 & 2 & 3 & 4 & 5 & 6 & 7 \\
\hline
Usage & PRT low & PRT high & Divisor & \multicolumn{4}{c |}{LCD info.} & 00h \\
\end{tabular}
\end{center}

the relevant section\footnote{Lines 471--519 of "synth.s"} of the network packet handler might look 
more like this:

\begin{h180}
# (MIDI note is already in L)
# Multiply by 8 to address 8-byte blocks
ld h, 0x08
mlt hl
# Perform lookup (offset from base address)
ld de, notedata_lookup
add hl, de
# Low PRT byte
ld a, (hl)
out0 (PRT0_RLD_L), a
# High PRT byte
inc hl
ld a, (hl)
out0 (PRT0_RLD_H), a
# Divisor
inc hl
ld a, (hl)
# Store the divisor in the shadow register (interrupts
# disabled to prevent double-swapping)
di
exx
ld c, a
exx
ei
# Set the location on the LCD
ld a, LCD_NOTE
call lcd_setlocation
# Output the note name
inc hl
call lcd_print
\end{h180}
% Merging the 2 MIDI lookup tables into one
