% vim: fo=aw2tq tw=100 spell
\chapter{Further Work}

\section{Note Lookup Tables}
Since the major use of the note lookup tables is by the network handler, which uses all values of 
both tables, given a little more time it would make sense to combine the playback and display 
information into a single lookup table.  The new lookup table would contain blocks of 8 bytes, 
however the data only amounts to 7 bytes.  This would actually be useful, as the 8th byte would be 
set to 00h, which is the string terminator used by "lcd_print"\footnote{Line 48 of "lcd.s"}.  If the 
new lookup table format was

\begin{nowordcount}
\begin{center}
\begin{tabular}{r || c | c | c | c | c | c | c | c}
Byte & 0 & 1 & 2 & 3 & 4 & 5 & 6 & 7 \\
\hline
Usage & PRT low & PRT high & Divisor & \multicolumn{4}{c |}{LCD info.} & 00h \\
\end{tabular}
\end{center}
\end{nowordcount}

the relevant section\footnote{Lines 471--519 of "synth.s"} of the network packet handler might look 
more like this:

\begin{nowordcount}
\begin{h180}
# (MIDI note is already in L)
# Multiply by 8 to address 8-byte blocks
ld h, 0x08
mlt hl
# Perform lookup (offset from base address)
ld de, notedata_lookup
add hl, de
# Low PRT byte
ld a, (hl)
out0 (PRT0_RLD_L), a
# High PRT byte
inc hl
ld a, (hl)
out0 (PRT0_RLD_H), a
# Divisor
inc hl
ld a, (hl)
# Store the divisor in the shadow register (interrupts
# disabled to prevent double-swapping)
di
exx
ld c, a
exx
ei
# Set the location on the LCD
ld a, LCD_NOTE
call lcd_setlocation
# Output the note name
inc hl
call lcd_print
\end{h180}
\end{nowordcount}


\section{Alternative Design}

During the late stages of the project I started development of an alternative design, which I 
abandoned due to time constraints.  The idea was that instead of the current software processing and 
output of samples from a wavetable, I would create hardware that could be programmed with an 
instrument number and note and would play the correct sound at the correct frequency.   This would 
leave the CPU with only network data and visual output to handle, but would also allow the extra CPU 
time to be used for other functions, such as an advanced visualisation on an liquid crystal matrix 
display.

My rudimentary ideas for this included the following:

\begin{itemize}
\item A ROM chip containing all of the wavetable samples (probably at 256 bytes per instrument).
\item An 8-bit counter connected to the lowest 8 bits of the wavetable ROM address for stepping 
through the instrument sample.
\item A 4-bit register connected to the next 4 bits of the wavetable address for selecting the 
instrument.
\item A 16-bit programmable reload timer running at a clock divider of 2, for controlling the 
playback frequency --- the output of this is the clock for the 8-bit address counter.  (Plus two 
8-bit registers to store the reload value.)
\item A note lookup table in software for the PRT values.
\end{itemize}

I built and experimented with the PRT section of this idea, and could successfully generate 
frequencies between 46.875Hz and 3.072MHz when using a system clock of 6.144MHz, by setting the 
reload value manually using two 8-bit DIP switches.

\section{Sound Reproduction}

There are two enhancements I would like to make to the system if I had the time:

\begin{itemize}
\item Implement percussion instruments.
\item Implement the ``attack'' portion of note playback.
\end{itemize}

Unfortunately, these would probably involve substantial restructuring of my playback.

\section{Notes}

\begin{itemize}
\item The wavetable for the ``piano'' instrument is missing from the submitted source code --- this 
was present during demonstration, but appears to have been accidentally lost during code cleanup.
\end{itemize}
