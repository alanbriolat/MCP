% vim: fo=aw2tq tw=100 spell
\chapter{Testing and Evaluation}
\label{sec:testing}

This section is a commentary on the testing procedures used throughout the project.  My general 
approach was bottom-up iterative development and testing---starting with the most basic components 
that stand on their own, testing them thoroughly, and building the next layer of complexity on top 
of them.  Every component that is tested well and known to be working provides a good stable 
platform for testing other components against.

\section{Output Hardware}
\label{sec:testing:hardware}

Using the principle of testing components with already working components, it made sense that the 
first part that needed to be tested was the audio hardware.

After assembling the audio output hardware as designed (and illustrated in 
Appendix~\ref{appendix:circuit-diagram}), but before connecting the data lines to the parallel I/O 
ports, the behaviour was tested manually by setting input values with 8-bit DIP switches, and 
measuring voltages with an oscilloscope.

First, both inputs were set to the maximum values, and the voltage measured from the output of the 
second op-amp (IC4, pin 6).  This is the maximum voltage.  Starting from the MSB, each bit of the 
input to the volume DAC was switched from ``on'' to ``off'', and the voltage observed at each point.  
The voltage was halved for every bit turned off, as expected.  This showed that the first DAC was 
operating correctly, and that the ``pre-scaling'' effect also worked correctly.

The experiment was repeated, but this time changing only the value of the waveform DAC.  The same 
results were observed, as expected.  This means this DAC was also working correctly.

Next, an 8-bit binary counter was assembled and run as the input to each DAC in turn to test the 
response to changing inputs.  The observed waveform on the oscilloscope revealed overshoot in the 
output when the value changed.  Fixing this issue is described in 
Section~\ref{sec:design:hardware:feedback}.  Other than this, the device performed as expected.

\todo{Not done yet: testing the audio amp}
